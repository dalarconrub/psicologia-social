% Options for packages loaded elsewhere
\PassOptionsToPackage{unicode}{hyperref}
\PassOptionsToPackage{hyphens}{url}
\documentclass[
]{book}
\usepackage{xcolor}
\usepackage{amsmath,amssymb}
\setcounter{secnumdepth}{5}
\usepackage{iftex}
\ifPDFTeX
  \usepackage[T1]{fontenc}
  \usepackage[utf8]{inputenc}
  \usepackage{textcomp} % provide euro and other symbols
\else % if luatex or xetex
  \usepackage{unicode-math} % this also loads fontspec
  \defaultfontfeatures{Scale=MatchLowercase}
  \defaultfontfeatures[\rmfamily]{Ligatures=TeX,Scale=1}
\fi
\usepackage{lmodern}
\ifPDFTeX\else
  % xetex/luatex font selection
\fi
% Use upquote if available, for straight quotes in verbatim environments
\IfFileExists{upquote.sty}{\usepackage{upquote}}{}
\IfFileExists{microtype.sty}{% use microtype if available
  \usepackage[]{microtype}
  \UseMicrotypeSet[protrusion]{basicmath} % disable protrusion for tt fonts
}{}
\makeatletter
\@ifundefined{KOMAClassName}{% if non-KOMA class
  \IfFileExists{parskip.sty}{%
    \usepackage{parskip}
  }{% else
    \setlength{\parindent}{0pt}
    \setlength{\parskip}{6pt plus 2pt minus 1pt}}
}{% if KOMA class
  \KOMAoptions{parskip=half}}
\makeatother
\usepackage{longtable,booktabs,array}
\usepackage{calc} % for calculating minipage widths
% Correct order of tables after \paragraph or \subparagraph
\usepackage{etoolbox}
\makeatletter
\patchcmd\longtable{\par}{\if@noskipsec\mbox{}\fi\par}{}{}
\makeatother
% Allow footnotes in longtable head/foot
\IfFileExists{footnotehyper.sty}{\usepackage{footnotehyper}}{\usepackage{footnote}}
\makesavenoteenv{longtable}
\usepackage{graphicx}
\makeatletter
\newsavebox\pandoc@box
\newcommand*\pandocbounded[1]{% scales image to fit in text height/width
  \sbox\pandoc@box{#1}%
  \Gscale@div\@tempa{\textheight}{\dimexpr\ht\pandoc@box+\dp\pandoc@box\relax}%
  \Gscale@div\@tempb{\linewidth}{\wd\pandoc@box}%
  \ifdim\@tempb\p@<\@tempa\p@\let\@tempa\@tempb\fi% select the smaller of both
  \ifdim\@tempa\p@<\p@\scalebox{\@tempa}{\usebox\pandoc@box}%
  \else\usebox{\pandoc@box}%
  \fi%
}
% Set default figure placement to htbp
\def\fps@figure{htbp}
\makeatother
\setlength{\emergencystretch}{3em} % prevent overfull lines
\providecommand{\tightlist}{%
  \setlength{\itemsep}{0pt}\setlength{\parskip}{0pt}}
\usepackage[]{natbib}
\bibliographystyle{plainnat}
\usepackage{booktabs}
\usepackage{bookmark}
\IfFileExists{xurl.sty}{\usepackage{xurl}}{} % add URL line breaks if available
\urlstyle{same}
\hypersetup{
  pdftitle={Psicología Social},
  hidelinks,
  pdfcreator={LaTeX via pandoc}}

\title{Psicología Social}
\author{}
\date{\vspace{-2.5em}2025-02-20}

\begin{document}
\maketitle

{
\setcounter{tocdepth}{1}
\tableofcontents
}
\chapter{\texorpdfstring{\href{https://dalarconrub.github.io/psicologia-social/}{Psicología Social}}{Psicología Social}}\label{psicologuxeda-social}

\textbf{Este resumen del tema es sólo para uso como material de apoyo en las tutorías. Para estudiar y preparar el examén se recomienda usar el libro referenciado en la guía docente de la asignatura.}

\chapter{\texorpdfstring{\href{https://dalarconrub.github.io/psicologia-social-tema-1/}{Tema 1: Introducción a la Psicología Social}}{Tema 1: Introducción a la Psicología Social}}\label{tema-1-introducciuxf3n-a-la-psicologuxeda-social}

\chapter{\texorpdfstring{\href{https://dalarconrub.github.io/psicologia-social-tema-2/}{Tema 2: Cognición Social}}{Tema 2: Cognición Social}}\label{tema-2-cogniciuxf3n-social}

\chapter{\texorpdfstring{\href{https://dalarconrub.github.io/psicologia-social-tema-3/}{Tema 3: Procesos de Atribución}}{Tema 3: Procesos de Atribución}}\label{tema-3-procesos-de-atribuciuxf3n}

\chapter{\texorpdfstring{\href{https://dalarconrub.github.io/psicologia-social-tema-4/}{Tema 4: Actitudes}}{Tema 4: Actitudes}}\label{tema-4-actitudes}

\chapter{\texorpdfstring{\href{https://dalarconrub.github.io/psicologia-social-tema-5/}{Tema 5: Estereotipos}}{Tema 5: Estereotipos}}\label{tema-5-estereotipos}

\chapter{\texorpdfstring{\href{https://dalarconrub.github.io/psicologia-social-tema-6/}{Tema 6: Conducta de ayuda}}{Tema 6: Conducta de ayuda}}\label{tema-6-conducta-de-ayuda}

\chapter{\texorpdfstring{\href{https://dalarconrub.github.io/psicologia-social-tema-8/}{Tema 8: Agresión}}{Tema 8: Agresión}}\label{tema-8-agresiuxf3n}

\chapter{\texorpdfstring{\href{https://dalarconrub.github.io/psicologia-social-tema-9/}{Tema 9: Autoconcepto e Identidad Social}}{Tema 9: Autoconcepto e Identidad Social}}\label{tema-9-autoconcepto-e-identidad-social}

\end{document}
